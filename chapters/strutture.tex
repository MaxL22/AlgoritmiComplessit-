% !TeX spellcheck = it_IT

\section{Strutture succinte, quasi-succinte e compatte}
Abbiamo lavorato con dei razzi, ora si parla di oggetti estremamente piccoli, strutture dati molto semplici.\\

Intanto, cos'è una struttura dati? Intendiamo un \textbf{Abstract Data Type ADT}, ovvero un \textbf{insieme di primitive}, ad esempio su uno stack le primitive potrebbero essere \texttt{push}, \texttt{pop} e \texttt{is\_empty}. Esistono modi formali, assiomatici, per definire le ADT.\\

Ogni ADT si può implementare e possono esserci molte \textbf{implementazioni} diverse, che funzionalmente corrispondono alla stessa ADT, ma differiscono per \textbf{complessità in spazio e tempo}.\\

\subsection{Information-Theoretical Lower Bound}

Supponendo di avere una struttura dati specifica che per \textbf{taglia} $n$ abbia $V_n$ \textbf{valori possibili}, che chiameremo: 
$$ v_1, v_2, \, \dots \, , v_{V_n} $$

Ed ognuno di questi possibili \textbf{valori} utilizza un \textbf{certo numero di bit}
$$ x_1, x_2, \, \dots \, , x_{V_n}$$

Allora il \textbf{teorema di Shannon} dice che
$$ \frac{x_1 + x_2 + \, \dots \, + x_{V_n}}{V_n} \geq \log_2 V_n $$

Sostanzialmente, la \textbf{media delle dimensione di tutti i valori} non può essere meglio del $\log_2$ del numero di valori possibili, quindi una "compressione", una rappresentazione dei dati non può essere meglio della media, ci saranno altre rappresentazioni per cui la rappresentazione occuperà più spazio.\\

\newpage

Questo $\log_2 V_n$ si chiama \textbf{information theoretical lower bound} e ci da la \textbf{dimensione media in bit per una struttura dati}.\\

Definiamo quindi 
$$ Z_n := \log_2 V_n$$
come il \textbf{numero di bit minimi per rappresentare una ADT}.\\

per una struttura dati che implementa l'ADT e occupa $D_n$ bit, allora 
$$ D_n \geq Z_n$$

Ci possono essere più casi: le strutture possono essere
\begin{itemize}
	\item \textbf{Implicite:} $D_n = Z_n + O(1)$, una costante in più del lower bound, e.g., $Z_n + 3$ bit.\\
	
	\item \textbf{Succinte:} $D_n = Z_n + o(Z_n)$, un $o$ in più del lower bound, e.g., $Z_n + \log Z_n$ bit.\\
	
	\item \textbf{Compatte:} $D_n = O(Z_n)$, un $O$ del lower bound, e.g., $Z_n + \sqrt{Z_n}$ bit.\\
	
	\item Tutte le altre, peggiori.\\
\end{itemize}

\newpage

\subsection{Struttura Succinta per Rank e Select}

Rank e Select ADT. Supponendo di avere un array di $n$ bit, $\underline b \in 2^n$.\\

Il \textbf{rank} è quanti 1 ci sono prima di una determinata posizione: 
$$ rank_{\underline b} (p) = |\{i | i<p \text{ e } b_i = 1\}| $$

\textbf{Select} è il duale, quindi dove si trova l'$i$-esimo 1 
$$ select_{\underline b} (k) = \max \{p | rank_{\underline b} (p) \leq k \}$$

Quindi, \textbf{rank} quanti 1 fino ad una posizione, mentre \textbf{select} dice dove solo gli 1.\\

Proprietà:
\begin{itemize}
	\item $rank(select(i)) = i$
	\item $select(rank(p)) \geq p$, viene $=$ se e solo se in posizione $p$ c'è un 1
\end{itemize}

Calcoliamo l'\textbf{information theoretical lower bound}. I valori possono essere 
$$ V_n = 2^n $$
in quanto questi sono i possibili valori di un vettore di $n$ bit. Di conseguenza 
$$ Z_n = \log_2 V_n = n $$
Per rappresentare un vettore di $n$ bit servono $n$ bit (e grazie al).\\

Per \textbf{fare rank e select} sul vettore possiamo: 
\begin{itemize}
	\item ad ogni richiesta calcolare la risposta, quindi occupiamo spazio $n$ (solo il vettore) ma le richieste richiedono tempo $O(n)$
	\item costruisco una tabella per tutti i possibili valori di rank e select, utilizzando spazio $2n \log n$ le richieste richiedono tempo $O(1)$ (solo il lookup)
\end{itemize}
 
\newpage

\subsubsection{Struttura succinta di Jacobson per Rank}

Si tratta di una \textbf{struttura multilivello}. \\

Il \textbf{vettore di bit} viene diviso in "\textbf{superblocchi}", i quali sono blocchi di lunghezza $(\log n)^2$ bit (si presuppone che $n$ sia potenza di due per evitare ceil vari).\\

Ogni superblocco viene \textbf{diviso in blocchi} di lunghezza $\frac{1}{2} \log n$.\\

Cosa memorizza Jacobson: 
\begin{enumerate}
	\item Per ogni \textbf{superblocco} memorizza quanti $1$ ci sono \textbf{prima del superblocco}.\\
	
	\item Per ogni \textbf{blocco} $B_{ij}$ memorizza quanti $1$ ci sono \textbf{dall'inizio del superblocco} $S_i$ fino al blocco $B_{ij}$ escluso.\\
	
	\item Four russians trick: una tabella rank per tutti i possibili valori dei blocchi
\end{enumerate}

\paragraph{Spazio:} Quanto \textbf{spazio} occupiamo per i \textbf{superblocchi}? Serve una \textbf{tabella per ogni superblocco} che indica gli 1:
\begin{itemize}
	\item ci sono $\frac{n}{(\log n)^2}$ \textbf{righe}
	\item per ogni riga ci possono essere $n$ valori, quindi servono $\log n$ bit per rappresentarli
\end{itemize}

In \textbf{totale}
$$ \frac{n}{(\log n)^2} \log n = \frac{n}{\log n} = o(n) $$

Per i superblocchi uso $o(n)$ bit aggiuntivi.\\

\newpage

Quanto \textbf{spazio per i blocchi}? Mi serve una tabella che indica gli 1 dall'inizio del superblocco: 
\begin{itemize}
	\item ci sono $\frac{n}{1/2 \log n}$ blocchi, di conseguenza altrettante \textbf{righe}
	\item per ogni riga devo scrivere quanti 1 dall'inizio del blocco, ovvero un numero che occuperà al massimo $\log ((\log n)^2)$ bit
\end{itemize}

In \textbf{totale} 
$$ \frac{n}{\frac{1}{2} \log n} 2 \log \log n = \frac{4 n \log \log n}{ \log n} = o(n) $$

Quindi per blocchi e superblocchi uso $o(n)$ bit aggiuntivi.\\

\paragraph{Four russians trick:} I blocchi sono lunghi $1/2 \log n$, quindi son corti, per rispondere a query su dati molto piccoli memorizzo tutte le possibili tabelle di rank e vado a cercare quella che mi serve (quella che corrisponde al blocco in questione). \\

Quanto \textbf{spazio} occupa? Ci sono 
\begin{itemize}
	\item $2^{1/2 \log n}$ possibili \textbf{blocchi}
	\item per ognuno dei quali ci sono $1/2 \log n$ \textbf{righe}
	\item per ogni riga servono $\log (1/2 \log n)$ bit
\end{itemize}

In \textbf{totale}
$$ 2^{\frac{1}{2} \log n} \cdot \frac{1}{2} \log n \cdot \log \left(\frac{1}{2} \log n\right) = \sqrt{n} \frac{1}{2} \log n \log \log \sqrt{n} = o(n) $$

Quindi \textbf{tutti i dati aggiuntivi} in totale occupano \textbf{spazio} $o(n)$.\\
Non possiamo tornare al vettore originale da questa struttura dati in quanto ci serve per fare il four russians trick.\\

Quindi abbiamo una \textbf{struttura succinta} che risponde in \textbf{tempo costante}, per avere un determinato rank prendiamo il rank prima del superblocco, lo sommiamo al rank prima del blocco e poi interroghiamo la tabella corrispondente al blocco all'interno del quale ci troviamo.\\

%End L19