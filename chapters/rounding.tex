% !TeX spellcheck = it_IT

\section{Tecniche basate sull'arrotondamento}
\addcontentsline{toc}{subsection}{\protect\numberline{}Programmazione Lineare}
\subsection*{Programmazione Lineare}

Serve per il prossimo algoritmo, bear with me.\\

La programmazione lineare ($LP$) può essere \textbf{descritta come problema di ottimizzazione}: 
\begin{itemize}
	\item \textbf{Input:} una \textbf{matrice di razionali} $A \in \mathbb{Q}^{m \times m}$, un \textbf{vettore} $b \in \mathbb{Q}^m$, \textbf{vettore} $c \in \mathbb{Q}^n$
	
	\item \textbf{Soluzioni ammissibili:} tutti i \textbf{vettori} $x \in \mathbb{Q}^n$ \textbf{tali che} $Ax \geq b$
	
	\item \textbf{Funzione obiettivo:} $c^T x$
	
	\item \textbf{Tipo:} minimizzazione $\min$
\end{itemize}

Si sa che $LP \in \mathcal{PO}$ (il più famoso è algoritmo di Karmarkar).\\

\addcontentsline{toc}{subsubsection}{\protect\numberline{}PL Intera}
\subsection*{PL Intera}
Programmazione lineare intera ($ILP$):
\begin{itemize}
	\item \textbf{Input:} matrice $A \in \mathbb{Q}^{m \times m}$, $b \in \mathbb{Q}^m$,  $c \in \mathbb{Q}^n$
	
	\item \textbf{Soluzioni ammissibili:} tutti i vettori $x \in \mathbb{Z}^n$ tali che $Ax \geq b$
	
	\item \textbf{Funzione obiettivo:} $c^T x$
	
	\item \textbf{Tipo:} minimizzazione $\min$
\end{itemize}

Cercando una\textbf{ soluzione intera al posto che razionale} il problema diventa $ILP \in \mathcal{NPO}c$.\\

\newpage

\subsection{VertexCover via Rounding}
Soluzione per VertexCover \textbf{basata sull'arrotondamento}.\\

\paragraph{Programmazione lineare per VertexCover:} Per il problema VertexCover $\pi$: Input: $G = (V,E)$, con pesi $w_i \in \mathbb{Q}^+$, $\forall i \in V$ (già definito prima, non riscritto). \\

Il problema di programmazione lineare intera associato a $\pi$ diventa $ILP(\pi)$: $x \in \mathbb{Z}^n$, ed i vincoli sono: 
$$
\begin{cases}
	x_i + x_j \geq 1 & \forall \{i,j\} \in E \\
	x_i \geq 0 & \forall i \in V \\
	x_i \leq 1 & \forall i \in V
\end{cases}
$$
La funzione \textbf{obiettivo da minimizzare} è:
$$ \sum_{i \in V} w_i x_i $$

I vincoli sostanzialmente dicono che il nodo è o 0 o 1, possono essere solo numeri interi, quindi il nodo è preso oppure no, ed il vincolo $\geq 1$ stabilisce che ogni arco è preso da almeno un nodo. \\
Ma questo rimane un problema $ \in \mathcal{NPO}c$.\\

La versione "rilassata" di $\pi$ è uguale ma \textbf{risolto con} $x \in \mathbb{Q}^n$ ($LP$ non $ILP$). Stessi vincoli, stessa funzione obiettivo.\\

Quindi, partiamo dal \textbf{problema di VertexCover} $\pi$, lo abbiamo trasformato in un problema di \textbf{programmazione lineare intera} $ILP(\pi)$, ci dimentichiamo della parte intera , \textbf{diventa} $LP(\pi)$, ed \textbf{arrotondiamo} a partire dalla soluzione ottima $x^\ast$ di $LP(\pi)$: 
$$ \pi = (V,E), w_i  \xrightarrow[]{} 
\begin{array}{c}
	ILP(\pi) \\
	w^\ast_{ILP} \\
\end{array}
\xrightarrow[]{}
\begin{array}{c}
	LP(\pi) \\
	w^\ast_{LP}
\end{array}
\xrightarrow{x^\ast}
\text{ Rounding }
\xrightarrow{}
\hat{x}
$$
$$ 
\hat{x} = \begin{cases}
	0 & \text{ se } x^\ast_i <  \frac{1}{2} \\ 
	1 & \text{ se } x^\ast_i \geq  \frac{1}{2} 
\end{cases}
$$

Chiamando $w^\ast_{ILP}$ la funzione obiettivo di $ILP(\pi)$ e $w^\ast_{LP}$ la funzione obiettivo di $LP(\pi)$.\\

\paragraph{Lemma 1:} $w^\ast_{LP} \leq w^\ast_{ILP}$.\\

\begin{proof}
	Il problema rilassato ha un superinsieme di soluzioni ammissibili, quindi la soluzione ottima sicuramente non può peggiorare, può solo essere migliore.\\
\end{proof}

\paragraph{Lemma 2:} $\hat{x}$ è una \textbf{soluzione ammissibile} di $ILP(\pi)$.\\

\begin{proof}
	Ricordando i vincoli
	$$\begin{cases}
		\hat{x}_i + \hat{x}_j \leq 1 & \forall \{i,j\} \in E \\
		0 \leq \hat{x}_i \leq 1 & \forall i \in V
	\end{cases}
	$$
	E la definizione di $\hat{x}_i$ (sopra).\\
	
	Inoltre sappiamo che $x^\ast_i$ rispetterà, in quanto soluzione ammissibile, gli stessi vincoli che sono presenti per $\hat{x}$.\\
	
	L'unico vincolo importante da verificare è che la somma sia $\geq 1$ e l'unico caso in cui può non succedere è quando entrambi sono $=0$ 
	$$ \hat{x}_i + \hat{x}_j \ngeq 1 \implies \hat{x}_i + \hat{x}_j = 0 \implies \hat{x}_i = 0, \; \hat{x}_j = 0  $$
	
	Ma questo succede solo quando gli $x_i^\ast$ erano entrambi $< 1/2$
	$$ x_i^\ast = 0, \; x_j^\ast = 0$$
	
	Ma questo è impossibile in quanto vorrebbe dire che:
	$$ x_i^\ast + x_j^\ast < 1 $$
	
	Il che è impossibile per i vincoli del problema $LP(\pi)$, risolto in maniera ottima.\\
\end{proof}

\newpage

\paragraph{Lemma 3:} $\forall i$, $\hat{x}_i \leq 2x_i^\ast$.\\

\begin{proof}
	Guardando i possibili casi: 
	$$ \hat{x}_i = 0 \implies x_i^\ast < 1/2 \implies 0 \leq 2 \cdot 1/2 $$
	$$ \hat{x}_i = 1 \implies x_i^\ast \geq 1/2 \implies 2x_i^\ast \geq 1 = \hat{x}_i $$
\end{proof}

\paragraph{Lemma 4:}
$$ w = \sum_i w_i \hat{x}_i \leq 2 \sum_i w_i x_i^\ast = 2 w_{LP}^\ast $$
Questo dice che la soluzione ottenuta dall'algoritmo di arrotondamento è $\leq$ (per il Lemma 3) di 2 volte la soluzione ottima del problema $LP(\pi)$.\\

Per il Lemma 1  per il Lemma 4, rispettivamente, possiamo vedere che:
$$ \frac{w}{w_{ILP}^\ast} \leq \frac{w}{w_{LP}^\ast} \leq \frac{2 w^\ast_{LP}}{w^\ast_{LP}} = 2$$
Quindi l'algoritmo di arrotondamento è una \textbf{2-Approssimazione di VertexCover}.\\

% End L8

\vfill

Sperimentalmente, l'approssimazione peggiora, sia per rounding che pricing VertexCover, con grafi sparsi. In qualsiasi caso sempre abbastanza meglio della 2-Approssimazione, generare input pessimi è difficile, solitamente sono casi estremi.\\

\newpage