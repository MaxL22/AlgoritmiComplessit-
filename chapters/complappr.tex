% !TeX spellcheck = it_IT

\section{Teoria della Complessità di Approssimazione}


%\subsection{Teorema PCP}
 
%Story time?

Torniamo nel mondo magico dei \textbf{problemi di decisione}. \\

Ricordiamo che si possono sempre esprimere come: conoscendo un \textbf{linguaggio} $L \subseteq 2^\ast$ (insieme di stringhe binarie), dobbiamo \textbf{stabilire se l'input} $x \in 2^\ast$ (stringa binaria) \textbf{appartiene a} $L$ (linguaggio), quindi stabilire se $x \in L$, tramite un \textbf{algoritmo di decisione}.\\

\subsection{MdT con Oracolo o Verificatore}

Con \textbf{MdT con Oracolo o verificatore} si intende una macchina di Turing nella quale l'input è sempre una stringa binaria $x \in 2^\ast$, ma è presente anche un "\textbf{oracolo}", il quale contiene un'\textbf{ulteriore stringa} $w \in 2^\ast$, a cui la MdT può accedere quando necessario. L'\textbf{output} sarà quindi una \textbf{decisione} (sì/no) e \textbf{dipende da input e stringa dell'oracolo}.\\

Per accedere all'oracolo la MdT scrive sul \textbf{nastro delle query}: scrivendo una posizione su questo nastro l'oracolo restituirà il bit presente in quella posizione sulla stringa dell'oracolo.\\

Possiamo pensare alle MdT con oracolo come se quest'ultimo fosse una "\textit{dimostrazione}" per l'input, per questo vengono anche chiamate \textbf{verificatori}.\\

\addcontentsline{toc}{subsubsection}{\protect\numberline{}Teorema}
\paragraph{Teorema:} un linguaggio $L \subseteq 2^\ast$ sta in $\NP$ se e solo se esiste una MdT $V$ con oracolo tale che: 
\begin{enumerate}
	\item $V(x,w)$ lavora in \textbf{tempo polinomiale} in $|x|$
	\item $\forall x \in 2^\ast$, $\exists w \in 2^\ast$ tale che $V(x,w) = $ sì, se e solo se $x \in L$. Se $x$ deve avere risposta "sì", esiste una $w$ che fa ottenere "sì"
\end{enumerate}

Equivalente ad una NDTM, al posto di dipendere da input e tutte le uscite, dipende da input e stringa di oracolo, il non determinismo viene introdotto in maniera differente ma sono equivalenti.\\

Quindi $\NP$ possiamo definirla come la \textbf{classi di problemi risolvibili da verificatori in tempo polinomiale}.

\newpage

\subsubsection{Verificatori Probabilistici}

Input e output come prima, ma il \textbf{verificatore probabilistico} ha \textbf{accesso} anche a una \textbf{sorgente di bit random} e risponde "sì" o "no" in base anche a questi. Si aggiunge una parte di randomness.\\

La macchina in sé è deterministica, ma il \textbf{risultato dipende da input, oracolo e bit casuali}.\\

Il verificatore probabilistico $V$ per il linguaggio $L$: 
\begin{enumerate}
	\item Lavora in \textbf{tempo polinomiale} per $|x|$.\\
	
	\item Se $x \in L$, allora $\exists w \in 2^\ast$ tale che $V(x,w)$ \textbf{accetta con probabilità 1}. Se una stringa è accettabile, deve esserci una dimostrazione/stringa $w$ dell'oracolo che la accetta indipendentemente dalla sorgente casuale.\\
	
	\item Se $x \notin L$, $\forall w \in 2^\ast$, $V(x,w)$ \textbf{rifiuta con probabilità} $\geq 1/2$. Se una stringa è da non accettare, nonostante i bit casuali ho una buona probabilità di non accettarla.\\
\end{enumerate}

\newpage

\subsection{Probabilistically Checkable Proof PCP}

Dai verificatori probabilistici viene il nome PCP: Probabilistically Checkable Proof.\\

Date due funzioni $r,q : \mathbb{N} \rightarrow \mathbb{N}$, \textbf{chiamo} 
$$ PCP [r,q] $$
La \textbf{classe dei linguaggi accettati da un verificatore probabilistico} che su input $x$ faccia $\leq q(|x|)$ query all'oracolo e $\leq r(|x|)$ letture di bit random.\\

Esempi di classi: 
\begin{itemize}
	\item Se non posso fare query o prendere bit random:
	$$ PCP [0,0] = \mathcal{P} $$
	La decisione dipende solo dall'input, quindi questa equivale a $\mathcal{P}$.\\
	
	\item Se posso fare query all'oracolo ma non posso accedere a i bit random
	$$ PCP [0, poly] = \NP $$
	Senza probabilità le "probabilità" diventano certezze, ovvero 0 o 1. Questa corrisponde a introdurre il non determinismo, ovvero coincide con $\mathcal{NP}$.\\
\end{itemize}

\paragraph{Teorema PCP [Arora, Safra, 1998]: } La classe $\NP$ equivale a:
$$\mathcal{NP} = PCP [O (\log n), O(1)] $$

Estraggo un \textbf{numero logaritmico di bit} e faccio un \textbf{numero costante di query all'oracolo}, introdurre un po' di randomness fa fare meno query all'oracolo. Inoltre la randomness aumenta logaritmicamente, al posto di avere una quantità polinomiale di letture all'oracolo, quindi la randomness è esponenzialmente "più potente" del non determinismo.\\

\newpage

Ad \textbf{esempio}: 
$$ SAT \in PCP [ 5 \log n + 7 \log \log n + 12, 157] $$

D'ora in poi facciamo riferimento a uno specifico verificatore $V$ che usa $r$ \textbf{bit random} e $q$ \textbf{query}
$$ V \in PCP [r,q] $$

Su \textbf{input} $x \in 2^\ast$:
\begin{itemize}
	\item Fa esattamente $q(|x|)$ \textbf{query}
	\item Estrae esattamente $r(|x|)$ \textbf{bit random}
\end{itemize}

Si può assumere questo \textbf{senza perdita di generalità}, al massimo potrebbe usarne di meno, ma non è una restrizione (posso fare quelle che mancano "a caso" e non usare, basta sia un upper bound).\\

Ci interessa il caso in cui le \textbf{query siano una costante} $O(1)$.\\

Inoltre possiamo assumere che i \textbf{bit random li estragga tutti assieme} come prima cosa, per poi tenerli da parte e usarli quando servono.\\

Considerando un \textbf{verificatore} 
$$ V \in PCP [r (n), q] $$
Dove $q$ è una costante mentre $r(n)$ è un verificatore $\in O(\log n)$. \\
Questo verificatore 
\begin{itemize}
	\item riceve input $w \in 2^\ast$
	\item Estrae $R \in 2^{r(|x|)} bit$
	\item Richiede all'oracolo la posizione $w_{i_1^{R,x}}$
	\item Può rispondere 0 o 1 e per ogni possibilità richiederà un numero $q$ di posizioni su $w$ e le \textbf{richieste effettuate} potrebbero essere diverse, \textbf{in base alle risposte precedenti potrebbero cambiare le richieste effettuate}
\end{itemize}

Si chiama \textbf{verificatore adattivo}, le \textbf{query} fatte all'oracolo \textbf{dipendono anche dalle risposte delle query precedenti}.  Sarebbe carino avere un verificatore non adattivo.\\

Quindi al posto di avere un albero di esecuzioni, guardiamo quali sono \textbf{tutte le possibili richieste}, le quali saranno una quantità finita
$$ \overline q = 2^{q-1}, 2^{q-2}, \, \dots \, , 1$$

Dopo aver estratto tutti i bit random che ci servono, \textbf{estraiamo} anche \textbf{tutte le possibili} $\overline q$ \textbf{query all'oracolo}. Dopo di questo l'esecuzione diventa quella di una MdT deterministica.\\

Questo ci permette di \textbf{trasformare qualunque verificatore adattivo} in un verificatore \textbf{non adattivo}.\\

%Quindi vedendo %?

Daremo per scontato che il verificatore: 
\begin{enumerate}
	\item \textbf{Legga} una \textbf{stringa} $R \in 2^{r(|x|)}$ di bit random
	\item \textbf{Effettui} un numero $q$ di \textbf{query} $\{i_1^{R,x}, \, \dots \, , i_q^{R,x}\}$
	\item Da qui il \textbf{comportamento} è \textbf{puramente deterministico}, dipendente da input $x$, stringa random $R$ e risposte ottenute dall'oracolo
\end{enumerate}

Se la stringa è da accettare, qualunque sia la stringa di bit random scelti, c'è una stringa dell'oracolo che mi fa accettare l'input. Se non è da accettare invece, per almeno metà delle stringhe di bit random viene rifiutato (probabilità $\geq 1/2$).\\

% End L17

\newpage

\paragraph{Esempio:} Prendendo 
$$ L \in PCP [r(n), q] = \NP $$
Come \textbf{funziona il verificatore}?
\begin{itemize}
	\item Prende in ingresso l'input 
	$$ z \in 2^\ast$$
	\item Estrae una stringa random
	$$ R \in 2^{r(|z|)} $$
	\item Scelgo le posizioni, che dipenderanno dall'input e da $R$, le quali saranno esattamente $q$
	$$ i_{1}^{z,R}, \, \dots \, , i_{q}^{z,R} \in \mathbb{N} $$
	che portano ad altrettante richieste all'oracolo e altrettanti bit: 
	$$ b_1, \, \dots \, , b_q $$
	\item Da qui in poi ci sarà da fare computazione deterministica, che dipende da:
	$$ z, R, b_1, \, \dots \, , b_q $$
\end{itemize} 

\newpage

\subsubsection{Inapprossimabilità di MaxEkSat}
Ogni $k$-CNF (con $k \geq 3$) con $t$ clausole si può \textbf{trasformare} in una 3-CNF con $(k-2)t$ clausole \textbf{preservando la soddisfacibilità}.\\

Esempio: 
$$ x_7 \vee \neg x_4 \vee x_9 \vee \neg x_{15} $$
$$ \implies (x_7 \vee \neg x_4 \vee y_1) \wedge (x_9 \vee \neg x_{15} \vee \neg y_1) $$

Se quella sopra è soddisfacibile, almeno uno dei letterali sopra è vero, l'assegnamento si può estendere a uno che rende vera anche quella sotto, mentre se tutti quelli sopra sono falsi, non c'è modo di avere la variabile ausiliaria $y_1, \neg y_1$ vera in entrambi i casi.\\

\addcontentsline{toc}{subsubsection}{\protect\numberline{}Teorema}
\paragraph{Teorema:} Per ogni $k \geq 3$, esiste un $\epsilon > 0$ tale che MaxEkSat non è $(1+\epsilon)$ approssimabile (supponendo che $\mathcal{P} \neq \mathcal{NP}$).\\

\begin{proof}
	(Per $k=3$). Scegliamo un $L \in \NP c$, quindi 
	$$ L \in \NP = PCP [r(n), q] $$
	per qualche $r(n) \in O(\log n)$, $q \in \mathbb{N}$.\\
	
	Sia $V$ il verificatore
	$$ z \in 2^\ast, \;\; R \in 2^{r(|z|)},\;\; i_1^{z,R}, \, \dots \, , i_q^{z,R} \rightarrow b_1, \, \dots \, ,b_q $$
	
	Si può scrivere una $q$-CNF soddisfacibile se e solo se $V$ accetta
	$$ \psi_{z,R} $$
	che dipende dalle variabili logiche $w_0, w_1, \, \dots$ dove $w_i = true$ se e solo se l'$i$-esimo carattere dell'oracolo è un 1 (corrispondono ai caratteri sulla stringa dell'oracolo).\\
	
	\newpage
	
	Quindi $\psi_{z,R}$ è una $q$-CNF con $\leq 2^q$ clausole.\\
	Può diventare $\varphi_{z,R}$, ovvero una 3-CNF con $\leq q 2^q$ clausole.\\
	
	Posso creare, in tempo polinomiale: 
	$$ \Phi_z = \bigwedge_{R \in 2^{r (|z|)}} \varphi_{z,R} $$
	3-CNF con $\leq q 2^q 2^{r (|z|)}$ clausole (un numero grosso, ma tutto polinomiale).\\
	
	Quindi dando in pasto questa formula $\Phi_z$ a un algoritmo (ipotetico, cerchiamo un assurdo) di approssimazione per MaxE3Sat con tasso di approssimazione $\alpha < 1+ \epsilon$, definendo
	$$ \epsilon = \frac{1}{2q2^q} $$
	
	Se $z \in L$, allora $\exists w$ che fa accettare $V$ con probabilità 1.\\
	Significa che $\Phi_z$ è soddisfacibile, quindi $q2^q2^{r(|z|)}$ clausole soddisfacibili.\\
	
	Se $z \notin L$, $\forall w$ allora $V$ rifiuta con probabilità $\geq 1/2$.\\
	Significa che il massimo numero di clausole soddisfacibili in $\Phi_z$ è 
	$$ \frac{q2^q2^{r(|z|)}}{2} + \frac{2^{r(|z|)}}{2} (q2^q - 1) 
	=  q2^q2^{r(|z|)} - \frac{2^{r(|z|)}}{2}
	$$
	
	Fornendo $\Phi_z$ a un algoritmo approssimato che permette di approssimare MaxE3Sat a meno di $1+\epsilon$ il numero di clausole che verranno soddisfatte sarà: 
	\begin{itemize}
		\item se $z \in L$
		$$ \geq \frac{q2^q 2^{r(|z|)}}{1 + \epsilon} $$
		\item se $z \notin L$
		$$ \leq q2^q2^{r(|z|)} - \frac{2^{r(|z|)}}{2} $$
	\end{itemize}
	
	\newpage
	
	Chiamando: 
	$$ A = q 2^q, \;\;  B = 2^{r(|z|)}, \;\; \epsilon = \frac{1}{2A} $$
	
	Dando $\Phi_z$ in pasto all'algoritmo MaxE3Sat approssimato, quindi fornirà un numero di clausole soddisfacibili: 
	\begin{itemize}
		\item se $z \in L$
		$$ \geq \frac{AB}{1 + \epsilon} $$
		\item se $z \notin L$
		$$ \leq AB - \frac{B}{2} $$
	\end{itemize}
	
	Si possono distinguere questi due casi? Ipotizzando che per assurdo:
	\begin{flalign*}
		AB - \frac{B}{2} > \frac{AB}{1 + \epsilon} & \implies \\
		& \implies AB + AB \epsilon - \frac{B}{2}  -\frac{B}{2}\epsilon - AB > 0 \\
		& \implies \epsilon \left(AB - \frac{B}{2}\right) - \frac{B}{2} > 0 \\
		& \implies \frac{1}{2A} \left(AB - \frac{B}{2}\right) - \frac{B}{2} > 0 \\
		& \implies \frac{B}{2} - \frac{B}{4A} - \frac{B}{2} > 0 \\
		& \implies - \frac{B}{4A} > 0 
	\end{flalign*}
	
	Ed è impossibile, quindi i due intervalli non si sovrappongono.\\
	In tempo polinomiale ho determinato l'appartenenza ad $L$, quindi il linguaggio originale, il che è impossibile (a meno che $\pnp$).\\
\end{proof}

Questo è il modo in cui si usa PCP, tipicamente la costruzione è, partendo da un problema $\NP c$, far vedere che si riesce a trasformare il comportamento del verificatore in una formula (fatto in tempo polinomiale), che data in pasto a un algoritmo che non esiste permette di confermare l'appartenenza al linguaggio.\\
Al variare del problema varia $q$, quindi varia anche $\epsilon$, dando una valutazione precisa di $q$ si potrebbe dare un numero ad $\epsilon$.\\

%End L18

\newpage

\subsection{Inapprossimabilità di MaxIndependentSet}
Il problema MaxIndependentSet consiste nel trovare l'insieme massimo di vertici indipendenti, ovvero senza lati tra di loro.\\

\textbf{Definizione}: 
\begin{itemize}
	\item \textbf{Input:} grafo non orientato $G=(V,E)$
	
	\item \textbf{Soluzioni ammissibili:} insiemi indipendenti (insiemi di vertici tali che non ci sono lati tra loro, contrario di clique)
	$$ X \subseteq V \; \text{ t.c. } \; \left(\begin{array}{c}
		X \\ 2 
	\end{array}\right) \cap E = \emptyset $$
	
	\item \textbf{Funzione obiettivo:} cardinalità dell'insieme di vertici, $|X|$
	
	\item \textbf{Tipo:} massimizzazione $\max$
\end{itemize}

\addcontentsline{toc}{subsubsection}{\protect\numberline{}Teorema}
\paragraph{Teorema:} Per ogni $\epsilon > 0$, MaxIndependentSet non è $(2-\epsilon)$ \textbf{approssimabile}.\\

\begin{proof}
	Consideriamo un linguaggio $L \in \NP c$, quindi $L$ è verificabile da 
	$$ L \in PCP [r(n), q] $$
	con $r(n) \in O(\log n)$, $q \in \mathbb{N}$.\\
	
	Fissiamo 
	$$ z \in 2^\ast $$
	Generiamo una sequenza di bit random e $q$ query all'oracolo
	$$ R \in 2^{r(|z|)} \;\;\;\;\; i_1^{z,R}, \, \dots \, , i_q^{z,R} \rightarrow b_1, \, \dots \, , b_q $$
	
	Tutto questo genera una $z$-configurazione, ovvero una coppia: 
	$$ (R, \{i_1^{z,R}: b_1, \, \dots \, , i_q^{z,R}: b_q\}) 	$$
	
	Per uno $z$ fissato, è composta da una serie di $2^{r(|z|)}$ bit random e le $q$ coppie bit interrogato-risposta. Ci sono tante $z$-configurazioni, uno per ogni possibile $R$ e per ogni possibile interrogazione.\\
	
	\newpage
	
	Costruisco un grafo non orientato $G_z$ dove: 
	\begin{itemize}
		\item i vertici sono le $z$-configurazioni accettanti.\\
		
		\item c'è un lato tra due configurazioni della forma
		$$ 
		(R, \{i_1^{z,R}: b_1, \, \dots \, , i_q^{z,R}: b_q\}), \;\;\;\;\; (R', \{i_1^{z,R'}: b_1', \, \dots \, , i_q^{z,R'}: b_q'\}) 
		$$
		
		se e solo se $R = R' \vee \exists k, k' \in \{1, \, \dots \, , q\}$ tali che $i_k^{z,R} = i_{k'}^{z,R'}$ ma $b_k \neq b_{k'}$. Ovvero, se le due stringhe random sono uguali oppure se ci sono due interrogazioni alla stessa posizione dell'oracolo che forniscono risposte diverse. \\
	\end{itemize}

	La relazione descritta dai lati si chiama "relazione di incompatibilità", due soluzioni collegate sono incompatibili.\\
	
	Questo è un grafone, i bit random sono in quantità $2^{r(|z|)}$ e le interrogazioni sono $2^q$, quindi $2^{r(|z|)} \cdot 2^q$ in totale, grosso ma comunque polinomiale.\\
	
	\paragraph{Fact 1:} Se $z \in L$, allora $G_z$ ha un insieme indipendente di dimensione $\geq 2^{r(|z|)}$.\\
	
	\begin{proof}
 		Se $z \in L$, allora $\exists \overline w$ che fa accettare con probabilità 1. Prendiamo tutte le configurazioni accettanti
		$$
		(R, \{i_1^{z,R}: b_1, \, \dots \, , i_q^{z,R}: b_q\})
		$$
		compatibili con $\overline w$.\\
		
		Proprietà:
		\begin{enumerate}
			\item questo insieme di vertici ha dimensione $\geq 2^{r(|z|)}$, in quanto qualunque sia $R$ devo accettare
			\item sono tutti non collegati da archi, ovvero indipendenti, in quanto se fossero collegate sarebbero incompatibili
		\end{enumerate}
		\nn
	\end{proof}
	
	\paragraph{Fact 2:} Se $z \notin L$, allora ogni insieme indipendente di $G_z$ ha cardinalità $\leq 2^{r(|z|) - 1}$.\\
	
	\begin{proof}
		Supponendo che $S$ sia un insieme indipendente di $G_z$, con cardinalità $> 2^{r(|z|) - 1}$. Questo $S$ è un insieme di configurazioni al cui interno non ci possono essere configurazioni incompatibili tra loro, quindi per ogni posizione richiesta da una qualsiasi configurazione con una qualsiasi $R$ risulta sempre lo stesso valore.\\
		
		Da $S$ si può costruire un $\overline w$ compatibile con tutti i vertici di $S$.\\
		$\implies \overline w$ viene accettato in tutte le configurazioni, quindi accetta con probabilità $> 1/2$, ma per definizione è impossibile.\\
	\end{proof}
	
	Ora diamo $G_z$ in pasto a un algoritmo approssimato per MaxIndependentSet con grado di approssimazione $< 2 - \epsilon$.\\
	I casi sono due
	\begin{itemize}
		\item $z \in L$: l'ottimo è $\geq 2^{r(|z|)}$
		\item $z \notin L$: l'ottimo $< 2^{r(|z|)}$
	\end{itemize}
	
	Il nostro algoritmo è approssimato quindi ottiene
	$$ \frac{2^{r(|z|)}}{2 - \epsilon} > 2^{r(|z|) - 1} = \frac{2^{r(|z|)}}{2} $$
	
	Comunque sempre maggiore di $2^{r(|z|)-1}$, ed è impossibile ottenerlo in tempo polinomiale (a meno che $\pnp$), quindi assurdo.\\
\end{proof}