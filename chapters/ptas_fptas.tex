% !TeX spellcheck = it_IT
\section{Schemi di approssimazione PTAS e FPTAS}
\subsection{PTAS per 2-LoadBalancing}

Ricordando cos'è un PTAS: problemi approssimabile a meno di una costante desiderata (arriva al tasso di approssimazione desiderato).\\

LoadBalancing già visto, quello delle macchine e task, bla bla bla. 2-LoadBalancing è la stessa cosa ma ha \textbf{solo 2 macchine}.\\

\textbf{Definizione} del problema:
\begin{itemize}
	\item \textbf{Input}: sequenza di $n$ task $t_0, \, \dots \, , t_{n-1}$
	\item \textbf{Soluzioni ammissibili}: assegnamenti dei task su 2 macchine, una funzione $\alpha: n \rightarrow 2$
	\item \textbf{Funzione obiettivo}: massimo carico tra le due macchine 
	$$ \max \left(\sum_{i: \alpha(i)=0} t_i , \sum_{i: \alpha(i)=0} t_a\right) $$
	\item \textbf{Tipo}: minimizzazione, $\min$
\end{itemize}

\nn

Per l'\textbf{algoritmo PTAS}: 
\begin{itemize}
	\item \textbf{Input}:  i task $t_0, \, \dots \, , t_{n-1}$ e un \textbf{costante di approssimazione} (circa) $\epsilon > 0$ (per avere una $1+\epsilon$-approssimazione)
\end{itemize}

\newpage

\paragraph{Passaggi:}
\begin{itemize}
	\item If $\epsilon \geq 1$: 
	\begin{itemize}
		\item Assegna tutti i task alla prima macchina 
		\item Termina
	\end{itemize}
	\nn
	
	\item Altrimenti, ordina i $t_i$ in \textbf{ordine decrescente}.\\
	
	\item \textbf{Fase 1}:
	\begin{itemize}
		\item \textbf{Calcola} $k \leftarrow \lceil \frac{1}{\epsilon} -1 \rceil$
		\item Si \textbf{cerca esaustivamente} l'assegnamento ottimo dei task $t_0, \, \dots \, , t_{k-1}$ (primi $k$ task, $2^k$ assegnamenti possibili)
	\end{itemize}
	\nn
	
	\item \textbf{Fase 2}:
	\begin{itemize}
		\item I \textbf{restanti task}  $t_k, \, \dots \, , t_{n-1}$ sono \textbf{assegnati} in modo \textbf{greedy} (sempre alla macchina più scarica)
	\end{itemize}
	\nn
\end{itemize}

L'assegnamento dei task "grossi" (i primi) viene risolto a forza bruta, il resto greedy.\\

\addcontentsline{toc}{subsubsection}{\protect\numberline{}Teorema}
\paragraph{Teorema:} l'algoritmo è una $(1+ \epsilon)$-approssimazione di $2$-LoadBalancing.\\

\begin{proof}
	Chiamando la \textbf{somma dei task}
	$$ T := \sum_{i \in n} t_i$$
	
	Se $\epsilon \geq 1$, otteniamo una 2-Approssimazione (butto tutto su una, di sicuro non può essere peggio di 2 volte la soluzione ottima).
	$$ L^\ast \geq \frac{T}{2}, \;\;\;\;\; L = T $$
	$$\implies \frac{L}{L^\ast} \leq \frac{T}{\sfrac{T}{2}}$$
	
	Nel caso in cui $\epsilon < 1$: alla \textbf{fine della prima fase} abbiamo l'\textbf{assegnamento ottimo} per i primi $k$ task e un carico sulle macchine 0 e 1, rispettivamente $y_0$ e $y_1$. \\
	
	Rimangono \textbf{da assegnare} i task $t_k, \, \dots \, , t_{n-1}$ per arrivare agli \textbf{assegnamenti finali} $L_0$ e $L_1$.\\
	
	Assumiamo (senza perdita di generalità) che $L_0 \geq L_1$ (\textbf{alla fine} la macchina con il \textbf{carico maggiore} è \textbf{la prima}).\\
	
	Due casi: 
	\begin{enumerate}
		\item $L_0 = y_0$, dopo la prima fase tutti i restanti task sono stati attribuiti alla seconda macchina.\\
		$\implies$ Abbiamo \textbf{ottenuto l'assegnamento ottimo}, per costruzione dell'algoritmo.\\
		
		\item Uno o più task di $t_k, \, \dots \, , t_{n-1}$ sono stati assegnati alla prima macchina, sia $t_h$ l'\textbf{ultimo assegnato} a suddetta macchina (il più piccolo). Di conseguenza
		$$ L_0 - t_h \leq L_1' \leq L_1$$
		
		(dove $L_1'$ è il carico della seconda macchina prima dell'assegnamento di $t_h$). Ovviamente prima di assegnare $t_h$ il carico sulla prima macchina era minore del carico sulla seconda macchina
		$$ \implies 2L_0 - t_h \leq L_0 + L_1 $$
		$$ T = L_0 + L_1$$
		$$ \implies L_0 - \frac{t_h}{2} \leq \frac{T}{2} $$
		$$ L_0 \leq \frac{T}{2} + \frac{t_h}{2} $$
		
		Ricordiamo che \textbf{i primi $k$ task sono} $\geq (k+1) t_h$, in quanto ordinati in ordine decrescente
		$$ T \geq (k+1) t_h $$
		
		I task dopo $t_h$ li minoriamo con 0.
		$$ \frac{T}{2} \geq (k+1) \frac{t_h}{2} $$
		
		\textbf{Quindi}
		$$ \frac{L_0}{L^\ast} \leq \frac{L_0}{\sfrac{T}{2}} \leq \frac{\frac{T}{2} + \frac{t_h}{2}}{\sfrac{T}{2}} = 1 + \frac{t_h}{T} \leq 1 + \frac{t_h}{(k+1) t_h} = 1 + \frac{1}{(k+1)} $$
		
		Ma \textbf{per definizione di} $k$ 
		$$ \frac{L_0}{L^\ast} \leq 1 + \frac{1}{(k+1)} \leq 1 + \frac{1}{\frac{1}{\epsilon} - 1 + 1} = 1 + \epsilon $$
	\end{enumerate}
	$$ \implies  \frac{L_0}{L^\ast} \leq 1 + \epsilon $$
\end{proof}

\addcontentsline{toc}{subsubsection}{\protect\numberline{}Teorema}
\paragraph{Teorema:} Il PTAS \textbf{richiede tempo} $O\left(n \log n + 2^{\min \left(\frac{1}{\epsilon}, n\right)}\right)$.\\

\begin{proof}
	$n \log n$ è il \textbf{tempo} per l'\textbf{ordinamento}.\\
	
	L'\textbf{esponenziale} è la \textbf{fase di assegnamento esatto}.\\
	
	Abbiamo $2^k$ assegnamenti ma 
	$$ 2^k = 2^{\lceil \frac{1}{\epsilon} - 1\rceil} \sim 2^{\frac{1}{\epsilon}}$$
	
	Ci sarebbe anche la parte di assegnamento greedy ma talmente breve che non conta.\\
\end{proof}

Quindi abbiamo un algoritmo PTAS, con un \textbf{tasso di approssimazione arbitrariamente preciso} ma con \textbf{tempo che peggiora esponenzialmente} in base a quest'ultimo.\\

\newpage

\subsection{Knapsack Problem}

I pirati arrivano in una grotta con degli oggetti, ognuno di questi $n$ oggetti ha un valore 
$$ v_0, \, \dots \, , v_{n-1} $$

ed i pirati vogliono portarsi a casa il maggior valore possibile, ma per portarli hanno solo uno zaino con una capacità $W$ e ogni oggetto ha un peso  
$$ w_0, \, \dots \, , w_{n-1} $$

I pirati vogliono portare a casa il maggior valore possibile con un peso che rientra nello zaino.\\

% End L11

\paragraph{Definizione del problema:}
\begin{itemize}
	\item \textbf{Input}: $n>0$ numero di oggetti, $v_i, w_i \in \mathbb{N}^+$ valori e pesi associati agli oggetti, $i <n$, capacità dello zaino $W>0$
	\item \textbf{Soluzioni ammissibili}: $X \subseteq n$ tale che 
	$$ \sum_{i \in X} w_i \leq W $$
	\item \textbf{Funzione obiettivo}: la somma dei valori 
	$$v = \sum_{i \in X} v_i$$
	\item \textbf{Tipo}: massimizzazione, $\max$
\end{itemize}

\newpage

\paragraph{Side quest: Programmazione Dinamica:} Vogliamo risolvere un certo problema $\pi$, nel quale sono presenti dei parametri $\pi [\overline{a},\overline{b}]$, ma il problema si può presentare in varianti con valori diversi da $\overline{a}$ e $\overline{b}$.\\

Voglio poter \textbf{risolvere} una certa \textbf{istanza di un problema a partire da istanze precedenti}/più semplici/con valori più semplici risolte in precedenza.\\
Devo quindi risolvere soluzioni più semplici per arrivare a quelle più grosse, le soluzioni dipendono da quelle prima.\\

Il numero di problemi da risolvere prima della soluzione che cerco è pari a tutte le possibili combinazioni di valori di parametri fino a $\overline{a}$ e $\overline{b}$. Si possono pensare le diverse istanze da risolvere come se fossero in una tabella con da un lato tutti i valori possibili del parametro $\overline{a}$ e dell'altro i possibili valori di $\overline{b}$.

$$
\begin{array}{c | c | c | c | c }
	& 0 & 1 & \dots & \overline{a} \\
	\hline
	0 & & & & \\
	\hline
	1 & & & & \\
	\hline
	\dots & & & & \\
	\hline
	\overline{b} & & & &
\end{array}
$$

In ognuna delle celle la soluzione dell'istanza di $\pi$ con i relativi valori dei parametri.\\

Fondamentale i parametri siano valori interi, altrimenti si può risolvere tutto in tempo polinomiale.\\

\paragraph{Side-side-quest: pseudopolinomialità:} Un algoritmo è pseudopolinomiale quando il suo \textbf{tempo di esecuzione dipende dal valore numerico degli input} considerati, al posto che dalla grandezza della rappresentazione. Un valore $n$ in binario richiede $b = \lceil \log_2 n \rceil$ bit, ma il valore considerato è $n \leq 2^b$, quindi il tempo d'esecuzione è esponenziale rispetto alla dimensione dell'input.\\

\newpage

\subsubsection{Soluzione di Programmazione Dinamica: Versione A}

Partendo dalle $n$ coppie di valori-pesi.\\

I parametri secondo cui iterare con la programmazione dinamica (righe e colonne della tabella) sono: 
\begin{itemize}
	\item $W$, \textbf{capienza dello zaino}, risolvo il problema per tutti i valori da 0 a $W$
	\item \textbf{Numero di oggetti}, considero solo i primi $i$ oggetti, variando il valore di $i$ tra 0 e $n$
\end{itemize}

Posso costruire una tabella con in \textbf{ogni riga tutte le soluzione per un possibile valore per la capienza dello zaino} $W$ e la colonna $i$ contiene le soluzioni considerando solo i primi $i$ oggetti. Quindi la cella $3,2$ contiene il valore massimo ottenibile con i primi $3$ oggetti e capienza $2$.\\

\textbf{Ogni cella} della tabella contiene il valore massimo che i pirati si portano a casa, i.e., il risultato del problema, i.e., il \textbf{valore della funzione obiettivo}.\\

La prima riga, non importa  quanti oggetti considero, se lo zaino ha capienza 0 non posso metterci nulla.\\
La prima colonna anch'essa è tutti zero, in quanto non ho oggetti, anche con un camion al posto dello zaino nella grotta non trovo nulla.\\

Quindi come posso trovare il \textbf{valore} $v(w,i)$ \textbf{per una qualsiasi cella?} Presupponendo di riempire la tabella da sinistra verso destra, dall'alto verso il basso. La $w$ rappresenta la capienza considerata, la funzione $v$ rappresenta il valore portato a casa e la $i$ indica che abbiamo considerato i primi $i$ oggetti, quelli che hanno da indice 0 a $i-1$.\\

Guardo \textbf{i valori precedenti}, partendo da $v(w,i-1)$:
\begin{itemize}
	\item Se decido di non aggiungere l'$i$-esimo elemento il risultato non cambia, in quanto ho già trovato il risultato migliore per quello stesso spazio
	\item Altrimenti, se voglio aggiungere l'$i$-esimo elemento con valore $v_i$, aggiungo il suo valore, ma lo zaino deve essere abbastanza capiente
\end{itemize}

Formalmente:
$$
v (w,i) = \begin{cases}
	v (w, i-1) & \text{ se } w < w_{i-1} \\
	\max \left(v(w,i-1), v_{i-1} + v (w - w_{i-1}, i-1)\right) & \text{ altrimenti }
\end{cases}
$$
Quindi, se non c'è abbastanza spazio ($w < w_{i-1}$) per aggiungere l'$i$-esimo elemento il valore rimane lo stesso.\\

Altrimenti prendo il massimo tra il valore della cella subito a sinistra e il valore della migliore cella la cui capienza associata mi permette di aggiungere l'$i$-esimo elemento.\\

Il numero di colonne è $n$, ma tutti i possibili valori di $W$ sono pseudo-polinomiali $\implies$ complessità esponenziale.\\

\newpage

\subsubsection{Soluzione di Programmazione Dinamica: Versione B}

Tengo la nostra tabella per la programmazione dinamica, in riga sempre il numero di oggetti considerati, ma \textbf{in colonna metto il valore che i pirati vogliono portarsi a casa}, di conseguenza \textbf{nelle celle} calcolo \textbf{la minima capacità di uno zaino per ottenere quel valore}. Stesso problema, stiamo solo cambiando la rappresentazione.\\

Se il valore obiettivo è 0, la dimensione necessaria è sempre 0, si tratta di un pirata molto umile.\\
Se voglio portare a casa un valore $>0$, ma sto considerando $0$ elementi risulta difficile $\implies$ la prima colonna è tutta $+ \infty$ (tranne la prima cella a zero).\\

Come ottengo \textbf{una cella qualunque} $w(v,i)$?\\

Se decido di non portare a casa anche l'$i$-esimo oggetto rimane tutto come $w(v, i-1)$, altrimenti diventa $w_{i-1} + w (v - v_{i-1}, i-1)$, quindi
$$ w (v,i) = \begin{cases}
	\min \left(w (v, i-1), w_{i-1}\right) & \text { se } v < v_{i-1} \\
	\min \left(w (v, i-1), w_{i-1} + w (v - v_{i-1}, i-1) \right) & \text{ altrimenti }
\end{cases}$$
Se il valore cercato è minore del valore dell'oggetto che ho appena introdotto posso scegliere di prendere solo quest'ultimo.\\

Altrimenti prendo il minimo tra il valore subito a sinistra e la somma tra il peso dell'$i$-esimo oggetto e il peso della cella contenente il valore minimo necessario per arrivare al nostro obiettivo una volta sommato il valore dell'oggetto appena inserito.\\

Quindi, compilo la tabella, ma \textbf{avrò delle celle in cui il peso minimo è too much}, $> W$, l'\textbf{output} per il nostro problema lo \textbf{ottengo risalendo} dalla cella in basso a destra verso l'alto \textbf{finché non trovo un valore di capienza minima ammissibile} $w < W$, i.e., prendo il valore ammissibile più alto possibile.\\

\newpage

\subsubsection{Algoritmo DP Approssimato}

Problema: gli algoritmi precedenti sono esatti, ma sono entrambi esponenziali nel numero di righe.\\

Per \textbf{approssimare il problema} possiamo "\textit{comprimere le righe}", ovvero \textbf{dividiamo tutti i valori per un certo numero}, \textbf{diminuendo} così il \textbf{numero di righe} necessarie, fino a far diventare l'algoritmo polinomiale.\\

Si può fare \textbf{solo con la seconda versione}, in cui le righe rappresentano la funzione obiettivo (che possiamo approssimare, andando a ottenere un sub-ottimo), mentre con la prima versione non possiamo comprimere la capacità dello zaino, in quanto parte dei constraint del problema, rischieremmo di andare verso soluzioni non ammissibili.\\

Quindi, applichiamo il secondo algoritmo, ma con \textbf{scaling}.\\

Al problema $\pi (v_i, w_i, W)$ aggiungiamo $\epsilon \in (0,1]$ e vogliamo una \textbf{$(1+\epsilon)$-approssimazione} di $\pi$
$$ \theta = \frac{\epsilon \cdot v_{max}}{2n}$$

Introduciamo \textbf{due ulteriori versioni del problema}: 
$$ \overline{\pi} \left(\overline{v}_i = \left\lceil \frac{v_i}{\theta} \right\rceil \theta, w_i, W\right) $$

Quindi arrotondo un po' i valori per eccesso, mentre in 
$$ \hat \pi = \left( \hat v_i = \left\lceil \frac{v_i}{\theta} \right\rceil, w_i, W \right)$$
approssimo ma lascio compresso, quindi noi vogliamo risolvere $\hat \pi$ in modo esatto tramite il secondo algoritmo di programmazione dinamica visto precedentemente.\\

\newpage

Dobbiamo capire \textbf{come sono correlate le tre soluzioni}:
$$ \hat X^\ast \subseteq n, \;\;\;\; \overline X^\ast \subseteq n \;\;\;\; X^\ast \subseteq n$$

\paragraph{Osservazione:} l'\textbf{ammissibilità è la stessa} per tutti e 3 i problemi, tutti i problemi hanno le stesse soluzioni ammissibili.\\

\paragraph{Osservazione:} Le soluzioni 
$$ \overline X^\ast = \hat X^\ast$$
in quanto sono problemi uguali, i valori sono solo moltiplicati per una costante. \textbf{Queste soluzioni sono uguali}.\\

\paragraph{Lemma:} Sia $X$ una soluzione ammissibile, 
$$ (1 + \epsilon) \sum_{i \in \hat X^\ast} v_i \geq \sum_{i \in X} v_i $$
La soluzione ottima dei nostri problemi modificati, moltiplicata per la costante di approssimazione, è sempre $\geq$ di qualsiasi soluzione ammissibile.\\

\begin{proof}
	Dato che $\overline \pi$ ha solo degli arrotondamenti per eccesso, tutti i valori saranno un po' più grandi, quindi 
	$$ \sum_{i \in X} v_i \leq \sum_{i \in X} \overline v_i \leq \sum_{i \in \overline X^\ast} \overline v_i $$
	E ovviamente, la soluzione ottima per $\hat \pi$ ha un valore maggiore di una qualsiasi soluzione ammissibile.\\
	
	Per come sono definiti 
	$$\overline v_i - v_i \leq \theta $$
	
	Quindi 
	$$ \sum_{i \in \overline X^\ast} \overline v_i 
	\leq \sum_{i \in \overline X^\ast} (v_i + \theta) 
	= \sum_{i \in \overline X^\ast} v_i + |\overline X^\ast| \theta \leq $$
	$$\leq \sum_{i \in \overline X^\ast} v_i + n \theta 
	= \sum_{i \in \overline X^\ast} v_i + n \frac{\epsilon \cdot v_{max}}{2n}
	$$ 
	
	Quindi 
	$$ \sum_{i \in X} v_i \leq \sum_{i \in \overline X^\ast} v_i + \frac{\epsilon \cdot v_{max}}{2} $$
	è vera per ogni soluzione ammissibile $X$.\\
	
	Prendo solo l'indice dell'elemento del valore massimo:
	$$ X = \{i_{max}\}$$
	(ovviamente escludendo gli elementi con $w_i>W$).\\
	
	Questa è una soluzione ammissibile, quindi
	$$ v_{max} \leq \sum_{i \in \overline X^\ast} v_i + \frac{\epsilon \cdot v_{max}}{2} \leq  $$
	Essendo $\epsilon \leq 1$
	$$\leq  \sum_{i \in \overline X^\ast} v_i + \frac{v_{max}}{2}$$
	$$ \implies \frac{v_{max}}{2} \leq \sum_{i \in \overline X^\ast} v_i $$
	
	Quindi la disuguaglianza precedente
	$$ \sum_{i \in X} v_i \leq \sum_{i \in \overline X^\ast} v_i + \frac{\epsilon \cdot v_{max}}{2} $$
	diventa
	$$ \sum_{i \in X} v_i \leq (1 + \epsilon) \sum_{i \in \overline X^\ast} v_i$$
	
	Ma ricordando che $\hat X^\ast = \overline X^\ast$
	$$ \sum_{i \in X} v_i \leq (1 + \epsilon) \sum_{i \in \hat X^\ast} v_i $$
\end{proof}

\newpage

\addcontentsline{toc}{subsubsection}{\protect\numberline{}Teorema}
\paragraph{Teorema:} Il lemma vale per tutte le soluzioni, anche quella originale
$$(1 + \epsilon) \hat v^\ast \theta \geq v^\ast$$

\begin{proof}
	Il Lemma dice che
	$$ (1 + \epsilon) \sum_{i \in \hat X^\ast} v_i \geq \sum_{i \in X} v_i $$
	Ma la prima parte è $\leq \hat v^\ast \theta$ e la seconda è una qualsiasi soluzione ammissibile, quindi anche $X^\ast$ e di conseguenza $v^\ast$.\\
\end{proof}

Comprimo i valori, calcolo tutto, ma alla fine devo \textbf{decomprimerli} per sapere il mio risultato, per questo ho il $\theta$ aggiunto, la discrepanza è data dall'approssimazione per eccesso effettuata.\\

\paragraph{Corollario:} L'algoritmo è una \textbf{$(1 + \epsilon)$-approssimazione}.\\

\paragraph{Complessità:} Devo guardare le celle della tabella, ovvero il numero di colonne per il numero di righe, quindi 
$$ n \cdot \sum \hat v_i \leq n^2 \hat v_{max} = n^2 \left\lceil \frac{v_{max}}{\theta} \right\rceil = $$
$$ = n^2 \left\lceil \frac{2 v_{max} n}{\epsilon \cdot v_{max}} \right\rceil = O\left(\frac{n^3}{\epsilon}\right) $$
la disuguaglianza è data dal fatto che ognuno degli $n$ elementi è per forza minore del massimo $v_{max}$

In totale quindi diventa
$$ O \left(\frac{n^3}{\epsilon}\right)$$
Ovvero, polinomiale in $n$ e in $\epsilon$, quindi \textbf{è un FPTAS}.\\

\newpage

%End L12